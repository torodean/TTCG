\chapter{Basic Rules}

\section{Relevant Terms and Definitions}
\begin{description}
    \item[CARD] Any valid card from the game.
    \item[DECK] The collection of cards a player uses in the game.
    \item[HAND] The cards a player holds from their deck.
    \item[DRAW] The act of taking a card from the deck and adding it to the hand.
    \item[SHUFFLE] The act of mixing a pile of cards.
    \item[DISCARD] The act of moving a card from the hand to the discard pile.
    \item[DISCARD PILE] The area where discarded, destroyed, or milled cards are placed.
    \item[SEARCH] The ability to look through a deck or discard pile for a specific card.
    \item[MILL] Sending cards from the top of the deck to the discard pile.
    \item[RECOVER] The ability to retrieve a card from the discard pile or another out-of-play area.
    \item[FIELD] The two-row area where cards are played by both players.
    \item[UNIT ZONE] The upper row of a player’s field, where unit cards are played.
    \item[SPELL ZONE] The lower row of a player’s field, where spell cards are played.
    \item[PLAY] The action of placing a card on the field in an active position.
    \item[ACTIVE] The top card of a stack on the field, considered in play and able to act.
    \item[TAP] Indicate a card has been used or activated by switching its position from vertical to horizontal.
    \item[UNTAP] Switching a cards position from horizontal to vertical (the opposing of 'tap').
    \item[RANK UP] The act of placing a card of one rank higher on top of another card (also called Level Up or Promote).
    \item[UNDER] Cards physically stacked beneath another card of the same type (e.g., unit under unit, spell under spell), often for layered effects.
    \item[BELOW] Spell cards in the lower row supporting a unit in the upper row of the two-row field.
    \item[DESTROY] The act of sending an active card to the discard pile.
    \item[TURN] The period when one player can play cards and take actions.
    \item[POINT] The unit tracking a player’s score, starting at 20.
    \item[UNIT] A card with Attack, Defense, and a non-spell type; rank determines its base stats (e.g., Rank 1: 500 total).
    \item[SPELL CARD] A card of type "spell," typically placed below units to provide effects and stat boosts.
    \item[TYPE] A primary card category: Water, Fire, Earth, Air, Nature, Electric, Light, Dark.
    \item[SUBTYPE] A card subcategory: Dragon, Warrior, Beast, Spell-caster, Flying, Insect, Reptile, Humanoid, Machine, Fairy, Ghost, Aquatic, etc.
    \item[RANK] The number of stars on a card, indicating its power level (also called Level).
    \item[ATTACK] The offensive stat of a unit card.
    \item[DEFENSE] The defensive stat of a unit card.
    \item[EFFECT] Text on a card that performs an action within the game.
    \item[ACTION] The act of performing a card’s effect.
    \item[EQUIP] The act of attaching a card to another for added effects.
    \item[COUNTER] The act of negating or modifying another effect or action.
    \item[IMMUNE] A status where a card is unaffected by specific effects.
    \item[TRIGGER] A condition that automatically activates an effect.
    \item[CHAIN] A method of resolving multiple effects in order.
    \item[INDESTRUCTIBLE] A status where a card cannot be destroyed (also called Immortal).
    \item[EXACTLY] A term requiring a condition to be precisely met for an effect to activate.
    \item[OWNING] The player whose deck a card originated from.
    \item[GAIN] The act of increasing a player’s points or a card’s stats.
    \item[LOSE] The act of decreasing a player’s points or a card’s stats.
    \item[MODIFY] The act of altering a card’s stats, rank, or type.
    \item[PHASE] A distinct segment of a turn (e.g., draw phase, play phase).
    \item[PLACE] The act of positioning a card in a specific location (e.g., under another card).
    \item[PREVENT] The act of stopping a specified action or effect from occurring.
    \item[ROW] One of the two horizontal areas on the field (upper for units, lower for spells).
    \item[STACK] A group of cards physically layered under another card of the same type (including the top card).
    \item[SWAP] The act of exchanging the positions or stats of two cards.
    \item[DORMATN] A card is considered dormant when it is physically under another card (not the top card of a stack).
\end{description}











\section{Basic Rules}

\subsection{Deck Construction}
\begin{itemize}
    \item Each player must construct a deck with a minimum of 50 cards and maximum of 70 cards.
    \item A player can have no more than 2 copies of a card with the same name (across all rarities) in their deck.
    \item A deck may contain any ratio of unit and spell cards.
\end{itemize}





\subsection{Game Start}
\begin{itemize}
    \item To decide who goes first, any random method can be used (e.g., coin toss, age, agreement). The default method for official play should be the roll of a die (assign one player as even, and another as odd).
    \item Each player begins the game by drawing 7 cards.
    \item If a player does not like their starting hand, they may shuffle it back into the deck and draw 6 new cards (mulligan). This can only be done once per game.
    \item The first player to play cannot attack on their first turn nor draw at the start of their turn.
\end{itemize}








\subsection{Turn Structure}
\begin{itemize}
\item Each player's turn consists of three phases:
	\begin{enumerate}
	    \item \textbf{Draw Phase}: At the start of every turn (except the first player's first turn), the player draws a card.
	    \item \textbf{Main Phase}:
	    \begin{itemize}
	        \item A player may play up to two cards to their field (unless a card specifies otherwise).	        
			\item Cards played or ranked up by effects must respect the unit-stack and spell-stack field limits.
	        \item A card effect that plays a card to the field does not count toward this limit.
	    \end{itemize}
	    \item \textbf{Battle Phase}: Combat and attacks occur.
	\end{enumerate}
\end{itemize}








\subsection{Playing Cards}
\begin{itemize}
    \item A card can only be played to the field in face up vertical position.
    \item When a card on the field activates an effect, the card is tapped (swapped to horizontal position).
    \item Tapping indicates one of the two effects has been used, preventing the other until untapped.
    \item \textbf{Spell Cards:}
    \begin{itemize}
        \item A Spell card effect can be used immediately after the card is played.
        \item Spell card effects can be activated at any point during either players turn once the card is on the field.
        \item If a spell card is activated in response to another action, a chain of effects occur.
    \end{itemize}
    \item \textbf{Unit Cards:}
    \begin{itemize}
        \item A unit card effect can be used immediately after the card is played.
        \item Unit effects can only be activated on the owning players turn.
    \end{itemize}
\end{itemize}






\subsection{Card Effects\label{CardEffectsRuleSection}}
\begin{itemize}
    \item Every card has two effects.
    \item All rules and effects are able to be overruled if a card effect explicitly states that it does so.
    \item Most effects can be activated at any time while the card is face-up on the field.
    \item If multiple effects trigger at the same time, they activate in reverse order of activation. If an effect is no longer valid because of this order, the effect simply does nothing (but still counts as activated).
    \item Tapped units cards cannot activate effects unless they specifically state otherwise.
    \item \textbf{Unit Cards:}
    \begin{itemize}
        \item When a unit effect is activated, that card is tapped immediately, then the effect activates.
        \item Tapped cards cannot activate effects.
        \item A unit card can attack whether tapped or not.
    \end{itemize}
    \item \textbf{Spell Cards:}
    \begin{itemize}
        \item When a spell card effect is activated (even if it's is a continuous effect), that card is tapped immediately, then the effect activates.
    \end{itemize}
    \item \textbf{Effect Conditions:}
    \begin{itemize}
        \item If a card effect activates when the card is sent to the discard pile or deck, it can only be activated if the card was initially untapped.
        \item Cards in the hand, deck, underneath other cards (dormant), and in the discard pile are considered untapped.
        \item A card effect cannot be activated when (in response to) it is destroyed (unless a card states otherwise).
    \end{itemize}
    \item \textbf{Effect Types:} 
    \begin{itemize}
        \item Various effects have different types which determine subtle differences in when and how they can be used.
        \item A list of the various effect types can be found in section \ref{CardEffectsSection}.
    \end{itemize}
\end{itemize}








\subsection{Leveling Up}
\begin{itemize}
    \item Any card with a rank higher than one can only be played to the field by ranking up another card of the same type.
    \item To rank up a card, place a card with the same type and one rank higher on top of that card.
    \item All cards under the card played are considered underneath that card and are immediately untapped. These are commonly called dormant cards.
    \item All dormant card remain on the field.
    \item Dormant cards cannot be tapped or have their effects activated, unless an effect states otherwise.
    \item dormant cards do not count towards the field unit limit.
\end{itemize}








\subsection{Battle Phase}
\begin{itemize}
    \item The last part of a players turn is the battle phase.
    \item Cards cannot be played to the field during the battle phase.
    \item Every unit a player owns can attack once during the owner players Battle Phase.
    \item If the opponent controls any units, the attacker must target one of them.
    \item If an opponent has no units remaining, a unit can attack the opponent directly.
    \item When a unit attacks, the attack value is used to determine its strength.
    \item When a unit is defending, the defense value is used to determine its strength.
    \item During an attack, whichever unit has the higher attack prevails, and the other is destroyed (unless a card states otherwise) and sent to the discard pile.
    \item When a card is destroyed in battle or by an effect, the dormant card directly underneath it becomes an active card.
\end{itemize}








\subsection{Gaining Points}
\begin{itemize}
    \item Every player starts the game with 20 points.
    \item To gain points, you must destroy units on your opponents side of the field.
    \item When a unit is destroyed, the owner of that unit loses points equal to the rank of that unit.
    \item During a players battle phase, if there are no units on the opponents side of the field, a unit can attack directly and the opposing player loses points equal to the rank of the attacking unit.
    \item Players can also gain or lose points based on various card effects.
\end{itemize}








\subsection{Field Limitations}
\begin{itemize}
    \item A players field is split into two parts (upper for units, and lower for spell cards).
    \item Dormant cards (cards under other cards within stacks) don’t count toward unit or spell limits.
    \item A players upper field can support a maximum of 5 unit card stacks.
    \item A players lower field can support a maximum of 5 spell card stacks.
\end{itemize}








\subsection{Hand Size Limit}
\begin{itemize}
    \item Players may hold a maximum of 10 cards in their hand.
    \item At the end of their turn, if a player has more than 10 cards, they must discard down to 10. This occurs after Battle Phase, before the next turn begins.
\end{itemize}







\subsection{Deck Exhaustion Rule}
\begin{itemize}
    \item If a player cannot draw a card because their deck is empty, they take 5 points of damage instead.
    \item This damage occurs at the start of their Draw Phase each turn that they are unable to draw.
\end{itemize}




