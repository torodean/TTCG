\chapter{Basic Rules}

\section{Relevant Terms and Definitions}
\begin{description}
    \item[CARD:] Any valid card from the game.
    \item[DECK:] The collection of cards a player plays with.
    \item[HAND:] The cards that the player has in their possession from their deck.
    \item[DRAW:] The act of taking a card from the deck and adding it to the hand.
    \item[SHUFFLE:] The act of mixing a pile of cards.
    \item[DISCARD:] The act of moving a card from the hand or field to the discard pile.
    \item[DISCARD PILE:] The area where used or destroyed cards are placed.
    \item[SEARCH:] The ability to look through a deck or discard pile for a specific card.
    \item[MILL:] Sending cards from the deck directly to the discard pile.
    \item[RECOVER:] The ability to retrieve a card from the discard pile or another out-of-play area.
    \item[FIELD:] The area where cards are placed by both players.
    \item[UNIT ZONE:] The top area of a player's field, where units cards are played.
    \item[SPELL ZONE:] The bottom area of a player's field, where spell cards are played.
    \item[PLAY:] The action of placing a card on the field.
    \item[TAP:] Switch the position of a card from vertical to horizontal or from horizontal to vertical.
    \item[LEVEL UP:] The act of placing a card of one level higher on top of another card.
    \item[UNDER:] Refers to all cards that are beneath another card (usually through leveling up cards).
    \item[DESTROY:] The act of sending a card and all cards under it to the discard pile.
    \item[TURN:] The period of time where only one player can play cards.
    \item[POINT:] The unit which keeps track of a player's score.
    \item[UNIT:] Any valid card from the game which has an attack, defense, and non-spell type.
    \item[SPELL CARD:] Any valid card from the game which has a type equal to spell.
    \item[TYPE:] One of the primary types of cards within the game - Water, Fire, Earth, Nature, Electric, Light, Dark.
    \item[SUB-TYPE:] One of the subtype categories of cards within the game - Dragon, Warrior, Beast, Spell-caster, Flying, Insect, Reptile, Humanoid, Machine, Fairy, Ghost, Fish, etc.
    \item[LEVEL:] The number of stars on a unit card.
    \item[ATTACK:] The attacking potential of a unit card.
    \item[DEFENSE:] The defensive potential of a unit card.
    \item[EFFECT:] The text written on a card which does some action within the game.
    \item[ACTION:] The act of performing an effect of a card.
    \item[EQUIP:] The act of attaching a card to another card for added effects.
    \item[COUNTER:] A response action that can negate or modify another action.
    \item[IMMUNE:] A status where a card is unaffected by specific effects.
    \item[TRIGGER:] A condition that activates an effect automatically.
    \item[CHAIN:] A method of resolving multiple effects in order.
    \item[DESTROY:] To send a card and all cards underneath it to the discard pile.
    \item[INDESTRUCTIBLE:]  This means a card cannot be \textit{destroyed}.
    \item[IMMORTAL:] This is an alternate term meaning a card is Indestructible.
    \item[EXACTLY:] This term is used when a condition has to be met for a card effect to activate.
    \item[RANK:] The number of stars on a card.
    \item[OWNING:] A player is considered the owner or owning player if the card on the field came from their deck.
\end{description}











\section{Basic Rules}

\subsection{Deck Construction}
\begin{itemize}
    \item Each player must construct a deck with a minimum of 50 cards and maximum of 70 cards.
    \item A player can have no more than 2 copies of the same card (across all rarities) in their deck.
    \item A deck may contain any ratio of unit and spell cards.
\end{itemize}





\subsection{Game Start}
\begin{itemize}
    \item To decide who goes first, any random method can be used (e.g., coin toss, age, agreement). The default method for official play should be the roll of a die (assign one player as even, and another as odd).
    \item Each player begins the game by drawing 7 cards.
    \item The first player to play cannot attack on their first turn nor draw at the start of their turn.
\end{itemize}








\subsection{Turn Structure}
\begin{itemize}
\item Each player's turn consists of three phases:
	\begin{enumerate}
	    \item \textbf{Draw Phase}: At the start of every turn (except the first player's first turn), the player draws a card.
	    \item \textbf{Main Phase}:
	    \begin{itemize}
	        \item A player may play up to two cards to their field (unless a card specifies otherwise).
	        \item A card effect that plays a card to the field does not count toward this limit.
	    \end{itemize}
	    \item \textbf{Battle Phase}: Combat and attacks occur.
	\end{enumerate}
\end{itemize}








\subsection{Playing Cards}
\begin{itemize}
    \item A card can only be played to the field in face up vertical position.
    \item \textbf{Spell Cards:}
    \begin{itemize}
        \item A Spell card effect can be used immediately after the card is played.
        \item Spell card effects can be activated at any point during either players turn once the card is on the field.
    \end{itemize}
    \item \textbf{Unit Cards:}
    \begin{itemize}
        \item A unit card effect can be used immediately after the card is played.
        \item Unit effects can only be activated on the owning players turn.
    \end{itemize}
\end{itemize}






\subsection{Card Effects}
\begin{itemize}
    \item Every card has two effects.
    \item All rules and effects are able to be overruled if a card effect explicitly states that it does so.
    \item Most effects can be activated at any time while the card is face-up on the field.
    \item If multiple effects trigger at the same time, they activate in reverse order of activation. If an effect is no longer valid because of this order, the effect simply does nothing (but still counts as activated).
    \item Tapped units cards cannot activate effects.
    \item \textbf{Unit Cards:}
    \begin{itemize}
        \item When a unit effect is activated, that card is then tapped.
        \item Tapped cards cannot activate effects.
        \item A unit card can attack whether tapped or not.
    \end{itemize}
    \item \textbf{Spell Cards:}
    \begin{itemize}
        \item When a spell card effect is activated (even if it's is a continuous effect), that card is tapped.
    \end{itemize}
    \item \textbf{Effect Conditions:}
    \begin{itemize}
        \item If a card effect activates when the card is sent to the discard pile or deck, it can only be activated if the card was initially untapped.
        \item Cards in the hand, deck, underneath other cards, and in the discard pile are considered untapped.
        \item A card effect cannot be activated when (in response to) it is destroyed (unless a card states otherwise).
    \end{itemize}
    \item \textbf{Effect Types:}
    \begin{itemize}
        \item Normal: A Normal effect is one that can be activated during your turn by tapping the card.
        \item Continuous: This effect is activated like a normal effect, but stays active while the card is on the field.
		\item Counter: This effect can be activated at any time in response to an opponents action.
		\item Equip: This card is 'equip' to another card and remains active until the other card or this card is destroyed. 
		\begin{itemize}
			\item When a card is destroyed, all equip cards connected to that card are destroyed with it.
			\item If an equip card is leveled up, it is no longer considered active or equip to what it was before leveling up.
		\end{itemize}
		\item Latent: This effect is triggered only when this card is currently untapped and meets some specific trigger condition. 
		\item Dormant: This is an effect that can only be activated when the card is tapped. These effects cannot be activated the same turn the card was tapped.
  		\item Passive: This effect is always active and is triggered whenever the effect conditions are met.
    \end{itemize}
\end{itemize}








\subsection{Leveling Up}
\begin{itemize}
    \item Any card with a level rank/level than one can only be played to the field by leveling up another card of the same type.
    \item To rank/level up a card, place a card with the same type and one rank higher on top of that card.
    \item All cards under the card played are considered underneath that card and are immediately untapped.
    \item All cards underneath another card remain on the field.
    \item Cards underneath another card cannot be tapped or have their effects activated.
    \item Cards underneath another do not count towards the field unit limit.
\end{itemize}








\subsection{Battle Phase}
\begin{itemize}
    \item The last part of a players turn is the battle phase.
    \item Every unit a player owns can attack once during the owner players Battle Phase.
    \item If the opponent controls any units, the attacker must target one of them.
    \item If an opponent has no units remaining, a unit can attack the opponent directly.
    \item When a unit attacks, the attack value is used to determine its strength.
    \item When a unit is defending, the defense value is used to determine its strength.
    \item During an attack, whichever unit has the higher power/attack prevails, and the other is destroyed (unless a card states otherwise) and sent to the discard pile.
\end{itemize}








\subsection{Gaining Points}
\begin{itemize}
    \item Every player starts the game with 20 points.
    \item To gain points, you must destroy units on your opponents side of the field.
    \item When a unit is destroyed, the owner of that unit loses points equal to the level of that unit.
    \item During a players battle phase, if there are no units on the opponents side of the field, a unit can attack directly and the opposing player loses points equal to the level of the attacking units.
    \item Players can also gain or lose points based on various card effects.
\end{itemize}








\subsection{Field Limitations}
\begin{itemize}
    \item A players field is split into two parts (upper for units, and lower for spell cards).
    \item A players upper field can support a maximum of 5 unit cards.
    \item A players lower field can support a maximum of 5 spell cards.
\end{itemize}








\subsection{Hand Size Limit}
\begin{itemize}
    \item Players may hold a maximum of 10 cards in their hand.
    \item At the end of their turn, if a player has more than 10 cards, they must discard down to 10.
\end{itemize}







\subsection{Deck Exhaustion Rule}
\begin{itemize}
    \item If a player cannot draw a card because their deck is empty, they take 5 points of damage instead.
    \item This damage occurs at the start of their Draw Phase each turn that they are unable to draw.
\end{itemize}




