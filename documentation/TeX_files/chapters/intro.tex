\chapter{Introduction and Concept Design}

A trading card game (TCG) is a popular game format where players acquire cards primarily through purchasing packs or trading with others. Players build customized decks from their collection, aiming to create synergistic combinations of cards that offer strategic balance. These decks serve as personalized tools for competitive play, where players test their strategies against opponents in dynamic, ever-evolving matches. As the name suggests, these are card games which should inherently emphasize trading between players. However, most TCGs in practice operate as product-driven systems where cards are primarily obtained through purchasing randomized packs or individual cards from retailers. The trading aspect is often secondary, with card value dictated by artificial scarcity, power creep, and secondary market speculation which essentially shifts the focus from player-driven exchange to a collectible gambling model.

This distinction between what a trading card game should be and what they typically are game me the idea for conceiving a \textit{true} TCG. This would be a card game where trade was emphesized, encouraged, and required in order for players to get the cards they need. Some of the major trade-limiting factors of modern games are as follows:
\begin{itemize}
	\item Many common cards have such a high abundance, that players who purchase booster packs have a large number of them, which makes trading for them unneccesary.
 	\item Key stable cards or cards of higher rarity are often limited and therefore worth a higher value than most other cards. This makes trades with them very uncommon.
  	\item The cards themselves are all valued based on their use in the current meta (strongest for battling) formats and decks. Certain cards are good for a short time and are therefore worth more, making people not likely to trade them. In contrast, many cards are deemed of less value when they are not in this status, making trading for them undesireable.
	\item Many TCGs sell pre-made decks or structure decks that remove the need to trade or collect cards organically, offering competitive-ready cards directly from the manufacturer.
 	\item Frequent set rotations or constant new releases incentivize continuous buying rather than fostering a stable, long-term trading ecosystem.
  	\item Companies intentionally design sets with a few highly desirable cards to drive pack sales, making acquiring cards through trading inefficient compared to buying.
   	\item Modern card rarities are pre-determined by the publisher and fixed across all packs, limiting the organic value assessment of cards by the player community.
    	\item Frequent reprints of popular cards by publishers devalue previously rare or sought-after cards, discouraging long-term trading and collecting.
     	\item High-end collector boxes with alternate art or foil versions, aimed at investors rather than players, further pushing the game toward a collectible model instead of a trading one.
      	\item The randomized nature of packs mirrors gambling systems, encouraging excessive spending with little guarantee of acquiring desired cards.
        \item Tournament entry prizes or exclusive promo cards available only through organized play events, creating a two-tiered system where certain cards are locked behind pay-to-participate systems.
\end{itemize}










\section{Other Popular Trading Card Games}

This is a very brief overview of some of the popular trading card games (TCG) with some pros and cons of each.







\subsection{Yu-Gi-Oh}

The Yu-Gi-Oh TCG was originally based on the anime series "Yu-Gi-Oh!", the card game was first released in 1999 by Konami in Japan. Players use a deck of cards to summon monsters, cast spells, and set traps to reduce their opponent's Life Points from 8000 to 0. The game is turn-based, with players drawing cards, summoning creatures, and activating effects to battle or control the field. The cards are split into three main types: monsters, which have attack and defense points for battling; spells, which offer various effects; and traps, which are hidden and activate under specific conditions. Summoning methods have expanded over time to include Normal Summoning, and more complex Special Summons like Fusion, Synchro, Xyz, Pendulum, and Link Summoning, each with its unique requirements. Deck construction involves a main deck with at least 40 cards, an Extra Deck for specialized monsters (up to 15 cards), and a Side Deck for between-duel adjustments (up to 15 cards). The game supports different formats like Advanced, which uses a banlist to manage overpowered cards, and Traditional, which is less restrictive. Strategy in Yu-Gi-Oh! centers around deck synergy, anticipating opponent moves, and managing resources effectively. The game has evolved through numerous card sets and mechanics, maintaining a dynamic meta-game where various deck archetypes cycle in and out of popularity, supported by a strong community and competitive scene including events like the World Championships.

\subsubsection{Issues}

Here are some commonly cited issues with the modern Yu-Gi-Oh! card game:
\begin{itemize}
	\item Complexity of Cards: Cards have become increasingly complex with lengthy text descriptions, leading to confusion and requiring players to interpret card effects, which can vary based on punctuation or syntax.
	\item Turn Length: Games often involve long turns where one player performs a series of actions that can take several minutes, reducing interactivity as the other player waits.
	\item Power Creep: There's a significant escalation in card power over time, making older cards obsolete and necessitating frequent deck updates to remain competitive.
	\item Special Summoning Mechanics: The introduction of multiple new summoning methods (like Synchro, Xyz, Pendulum, and Link Summoning) adds layers of complexity that can be overwhelming for new or returning players.
	\item Hand Traps and Negation: The prevalence of hand traps and negation effects can make games feel one-sided, where one player might not get to play effectively if they don't have the right responses in hand.
	\item Game Balance: There's a perception that the game lacks balance due to the continuous introduction of new, powerful cards without effective rotation or a robust banlist that addresses all issues.
	\item Cost and Accessibility: The game can be expensive to stay competitive due to the rarity of key cards and the need to constantly update decks. This also affects accessibility for new players. 
	\item Archetype Dependency: The modern game heavily revolves around archetypes or "series" of cards that have specific names or sub-types, which are designed to work synergistically with each other. This means players are often forced into investing in a narrow selection of cards to build a competitive deck, ignoring the vast majority of cards that don't fit into these specific archetypes. This can lead to:
		\begin{itemize}
			  \item Reduced Deck Diversity: Players are less inclined to experiment with a wide variety of cards since non-archetype cards might not provide the same level of strategic depth or interaction, leading to repetitive gameplay centered around a few dominant decks.
			  \item Barrier for New Players: For newcomers, the game can seem daunting not just because of the mechanics but also because of the need to invest in very specific cards to compete, which can be overwhelming both financially and in terms of game knowledge.
		  \end{itemize}
	\item Rarity and Price: Key cards, especially those that define or enhance archetype decks, are often released in high rarity, which significantly increases their cost. This makes keeping up with the meta-game expensive, especially as new sets introduce cards that can shift the competitive landscape.
	\item Lack of Official Rulings Database: Without an official database for card rulings, players and judges must interpret card text during tournaments, leading to potential inconsistencies and disputes.
	\item Short Game Duration with High Complexity: Matches can end quickly (sometimes in 1-2 turns), yet the complexity within these turns can be daunting, not matching the entertainment value of longer, strategic games.
	\item Format Issues: There's no official alternative format supported by Konami that would cater to different play styles or levels of complexity, potentially alienating players who prefer a less intense gameplay experience.
\end{itemize}

These issues are often discussed by the community, with various opinions on how much they impact the enjoyment of the game or its competitive scene. Some players appreciate the strategic depth these elements add, while others find them to be barriers to entry or fun.

\subsubsection{Positives}

Here are some of the positives of the Yu-Gi-Oh! trading card game, including aspects that were particularly beneficial before some of the more recent issues became prominent:

\begin{itemize}
\item Strategic Depth: Yu-Gi-Oh! offers a high level of strategic complexity. Players must think several moves ahead, manage resources, anticipate their opponent's strategies, and adapt on the fly, which can be intellectually satisfying.
\item Deck Building Creativity: Before the archetype system became so central, players had more freedom to experiment with deck building. Even now, there's still room for creativity within archetypes or for players who enjoy building rogue decks.
\item Community and Social Interaction: The game fosters a strong community through local events, online forums, and global tournaments. It's a social experience that brings people together, encouraging camaraderie and competition.
\item Variety of Play Styles: There's a wide range of deck archetypes and strategies, from control to combo, from beatdown to stall, allowing players to find a style that suits their personality or play preference.
\item Constant Evolution: The introduction of new cards and sets keeps the game dynamic. While this can contribute to some issues, it also means the game never gets stale, offering new strategies and mechanics to explore.
\item Accessibility to Casual Play: For those not interested in the competitive scene, Yu-Gi-Oh! can still be enjoyed casually with friends or at local game stores where the pressure to keep up with the meta isn't as intense.
\item Affordable Entry Point: While the competitive side can be expensive, the base game is accessible with structure decks or starter sets that provide a playable experience at a relatively low cost, especially for those just looking to enjoy the game casually.
\item Cultural Impact: The game has a significant cultural footprint, with the anime, manga, and various adaptations contributing to a rich lore and fanbase, which adds to the enjoyment of playing.
\item Educational Benefits: Playing Yu-Gi-Oh! can help with math skills (calculating damage, resource management), reading comprehension (understanding complex card text), and strategic thinking.
\item Replayability: Due to the randomness of draws and the myriad of possible interactions between cards, no two games are exactly alike, providing high replayability.
\item Global Events: The World Championships and other major tournaments give players something to strive for, offering not just competitive play but also a sense of achievement and community recognition.
\item Art and Design: The artwork on Yu-Gi-Oh! cards is often praised for its quality and thematic richness, enhancing the aesthetic appeal of collecting and playing with the cards.
\end{itemize}

Before some of the modern complexities and cost issues became as pronounced, these positives made Yu-Gi-Oh! appealing to a broad audience, from casual players enjoying the thematic elements and simple battles to competitive players who thrived on its strategic depth. Even with its challenges, many of these positives still hold true for the game today.














\subsection{Magic The Gathering}

Magic: The Gathering (MTG), created by Richard Garfield and first published by Wizards of the Coast in 1993, is a pioneering trading card game where players assume the roles of wizards, casting spells, invoking creatures, and using artifacts to battle each other. In MTG, each player starts with a life total of 20 and aims to reduce their opponent's life to zero or achieve an alternative win condition like decking (making the opponent draw from an empty deck). The game uses a deck of at least 60 cards, divided into several card types: Lands for mana generation, Creatures for combat, Sorceries and Instants for one-time effects, Enchantments for ongoing effects, Artifacts for various utilities, and Planeswalkers for dynamic, player-like abilities. MTG's gameplay involves strategic deck construction and resource management, where players must balance the use of their mana to play cards at the right time. The game's depth comes from its vast card pool, multiple formats (like Standard, Modern, Legacy, and Commander), and the continuous introduction of new sets, which bring fresh mechanics and themes. Magic: The Gathering is renowned for its community, with organized play through local game stores, regional and international tournaments, and the Pro Tour. Its rich lore, intricate artwork, and the ability to blend strategy with storytelling have made MTG not only a competitive game but also a cultural phenomenon in the world of tabletop gaming.

\subsubsection{Issues}

Here are some commonly cited issues with the modern Magic: The Gathering card game:

\begin{itemize}
\item Complexity of Cards: Modern MTG cards often feature dense text with intricate mechanics (e.g., double-faced cards, mutate, companion rules), which can confuse new players or lead to rules disputes, especially with interactions that require precise timing or keyword understanding.
\item Power Creep: Over time, card power levels have escalated (e.g., pushed rares in recent sets like Modern Horizons), rendering older cards less viable in competitive formats and pressuring players to constantly acquire new staples.
\item Game Pace and Interaction: Formats like Commander can have excessively long turns due to combo-heavy decks or board states, while faster formats like Modern or Standard can end abruptly with little back-and-forth if one player draws the right answers or threats.
\item Mana System Frustrations: The land-based resource system, while iconic, can lead to non-games due to mana flood (too many lands) or mana screw (too few), reducing player agency compared to games with less variance in resource access.
\item Format Fragmentation: With numerous formats (Standard, Modern, Legacy, Commander, Pauper, etc.), players can feel overwhelmed choosing where to invest, and some formats (e.g., Legacy) have high barriers due to the Reserved List inflating card prices.
\item Cost and Accessibility: Competitive play demands expensive decks, with staples like fetchlands, shocklands, or chase mythics (e.g., Ragavan, Nimble Pilferer) costing hundreds of dollars. Even casual Commander decks can spiral in price with desirable staples.
\item Set Fatigue: The rapid release schedule (multiple sets per year, plus Secret Lairs and supplemental products) makes it hard for players to keep up financially and mentally, diluting focus and overwhelming collectors.
\item Color Identity Dependency: Decks heavily rely on adhering to specific color identities or archetypes (e.g., blue for control, green for ramp), which can limit creative deck-building outside of established strategies and push players toward "solved" metas.
	\begin{itemize}
		\item Reduced Deck Diversity: In competitive formats, the meta often coalesces around a handful of dominant decks (e.g., Izzet Murktide in Modern), discouraging experimentation with off-meta strategies.
		\item Barrier for New Players: Newcomers face a steep learning curve not just with rules but also with understanding which cards are format-defining, requiring significant investment to catch up.
	\end{itemize}
\item Rarity and Price: Key cards are often printed at mythic rarity or in limited-run products (e.g., Secret Lairs), driving up costs and creating a pay-to-win perception, especially as reprints lag behind demand.
\item Rulings Ambiguity: While MTG has a comprehensive rules database, niche interactions (especially in Commander or with older cards) can still spark debates, relying on judge calls or community consensus.
\item Short Game Duration in Some Formats: Formats like Vintage or cEDH (competitive Commander) can end in 1-3 turns due to hyper-efficient combos, clashing with the expectation of interactive, strategic play.
\item Lack of Casual Support: Official support leans heavily toward competitive play or Commander, leaving casual players without a clear, low-stakes format tailored to simpler gameplay or kitchen-table magic.
\end{itemize}

These issues are debated within the MTG community, with some players embracing the complexity and others feeling it alienates newcomers or shifts focus too heavily toward profit-driven design.

\subsubsection{Positives}

Here are some of the positives of Magic: The Gathering, including aspects that shine in its current state and historically:

\begin{itemize}
\item Strategic Depth: MTG’s blend of resource management, timing, and bluffing creates a rich strategic experience. Players must balance offense, defense, and deck synergy, rewarding skill and foresight.
\item Deck-Building Freedom: While archetypes exist, MTG’s color pie and mana system allow for diverse deck construction. From janky brews to tier-one lists, players can express creativity across formats.
\item Community and Social Bonds: MTG thrives on its global community, with local game store events (e.g., Friday Night Magic), online play via MTG Arena, and massive tournaments fostering connection and rivalry.
\item Variety of Play Styles: Formats cater to different preferences—Standard for a rotating meta, Commander for multiplayer chaos, Legacy for eternal power, Pauper for budget brews—ensuring something for everyone.
\item Constant Evolution: New sets introduce fresh mechanics (e.g., Duskmourn’s eerie themes or Bloomburrow’s creature focus), keeping the game dynamic while building on its 30+ year history.
\item Casual Accessibility: MTG Arena offers a free-to-play entry point, and physical starter decks or preconstructed Commander decks provide affordable ways to dip into the game casually.
\item Affordable Entry Point: While competitive play is costly, casual options like Jumpstart packs or precons keep the barrier low for new or returning players looking to have fun without breaking the bank.
\item Cultural Legacy: As the first trading card game, MTG’s lore—spanning planes like Ravnica, Innistrad, and Zendikar—adds depth, enhanced by novels, art, and a growing media presence (e.g., Netflix series rumors).
\item Educational Value: MTG sharpens critical thinking (e.g., sequencing plays), math skills (e.g., combat calculations), and adaptability, often praised as a mental workout disguised as fun.
\item Replayability: Randomized draws, diverse matchups, and format-specific metas ensure games rarely feel identical, offering endless replay value.
\item Global Events: Pro Tours, MagicCons, and Commander-focused gatherings give players aspirational goals, blending competition with celebration of the game’s culture.
\item Art and Design: MTG’s card art is a hallmark, with stunning illustrations (e.g., Rebecca Guay’s classics or John Avon’s lands) and thematic cohesion that elevate collecting and gameplay.
\end{itemize}

Historically, MTG’s elegant core design and slower pace made it a standout, and even with modern challenges, its adaptability and community passion keep it a titan in the TCG world.












\subsection{Pokemon}

The Pokémon Trading Card Game (Pokémon TCG), launched in 1996 by Creatures Inc. and managed by The Pokémon Company, is a collectible card game tied to the iconic Pokémon franchise, blending simple mechanics with nostalgic appeal and strategic depth. Players build decks from a vast pool of Pokémon, Trainer, and Energy cards, aiming to knock out opponents’ Pokémon and claim 6 Prize cards to win, with gameplay centered on attaching energy to power attacks and evolving Pokémon through stages. Its accessibility—rooted in straightforward rules and affordable starter decks—makes it a gateway for younger players and casual fans, while modern expansions like Scarlet \& Violet introduce mechanics (e.g., Tera Pokémon, VMAX) that cater to competitive scenes. Backed by stunning card art, a global community, and events like the Pokémon World Championships, the TCG thrives on its franchise synergy, though it faces critique for power creep, archetype dominance, and high costs for meta cards, balancing simplicity with a dynamic, evolving meta as of February 21, 2025.

\subsubsection{Issues}

Here are some commonly cited issues with the modern Pokémon Trading Card Game:

\begin{itemize}
\item Simplistic Core Mechanics: While accessible, the game’s reliance on basic attack-damage mechanics and energy attachment can feel less strategic compared to the layered complexity of competitors like Yu-Gi-Oh! or MTG, limiting depth for advanced players.
\item Power Creep: Newer Pokémon (e.g., Pokémon V, VMAX, and ex cards) consistently outclass older ones with higher HP and damage output, making older decks obsolete and requiring frequent updates to stay competitive.
\item Turn Length and Passivity: Early turns can be slow as players build up energy and evolve Pokémon, leading to passive gameplay where one player waits for the other to “set up,” especially in mirror matches or against stall decks.
\item Prize Card System: The prize card mechanic (winning by taking 6 prizes) can lead to swingy games where drawing a key card early or late disproportionately decides the outcome, reducing skill expression in some cases.
\item Item and Supporter Dependency: Competitive decks heavily rely on specific Trainer cards (e.g., Boss’s Orders, Professor’s Research) for consistency and disruption, creating a bottleneck where games hinge on drawing these staples rather than Pokémon interactions.
\item Cost and Accessibility: Meta-defining cards (e.g., rare Alternate Arts or Secret Rares) are expensive due to collector demand, and the lack of affordable reprints makes building top-tier decks costly, despite the game’s kid-friendly image.
\item Archetype Dominance: The game pushes specific Pokémon types or strategies (e.g., Charizard, Gardevoir) through synergistic support cards, limiting deck diversity and pressuring players to invest in a narrow pool of viable archetypes.
	\begin{itemize}
		\item Reduced Deck Diversity: Tournaments often feature a handful of dominant decks (e.g., Lugia VSTAR, Miraidon ex), discouraging experimentation with rogue strategies or less-supported Pokémon.
		\item Barrier for New Players: Newcomers face not only rule learning but also the need to acquire specific, often pricey cards to compete, overwhelming those without prior knowledge or budget.
	\end{itemize}
\item Rarity and Price: Chase cards like full-art Trainers or hyper-rare Pokémon are printed at low rates, inflating secondary market prices and tying competitive success to financial investment.
\item Lack of Official Rulings Clarity: While simpler than some TCGs, edge-case interactions (e.g., Ability timing) lack a robust, centralized rulings database, leading to occasional confusion at events.
\item Short Competitive Games: Top-tier decks can close out games in 3-5 turns, especially with fast setups like one-Prize attackers (e.g., Radiant Charizard), clashing with the expectation of evolving Pokémon over time.
\item Format Rotation Issues: Standard rotation refreshes the meta but can alienate players who lose access to favorite cards, and the lack of a widely supported eternal format limits options for those wanting to use older collections.
\end{itemize}

These issues are often debated within the Pokémon TCG community, with some enjoying its simplicity and others wishing for more strategic nuance or affordability.

\subsubsection{Positives}

Here are some of the positives of the Pokémon Trading Card Game, including aspects that shine today and historically:

\begin{itemize}
\item Accessibility and Simplicity: The game’s straightforward rules—attach energy, evolve Pokémon, attack—make it easy for beginners, kids, and casual players to pick up and enjoy without a steep learning curve.
\item Deck-Building Creativity: While archetypes dominate competitive play, casual players can experiment with a wide range of Pokémon, Trainers, and strategies, especially with the variety of Pokémon types and evolution lines.
\item Community and Social Appeal: Local leagues, online play via Pokémon TCG Live, and global events foster a welcoming community, blending nostalgia with competition for players of all ages.
\item Variety of Play Styles: Decks range from aggressive single-Prize attackers to tanky VMAX stall builds to combo-heavy evolution strategies, offering options for different preferences.
\item Constant Evolution: New sets (e.g., Scarlet \& Violet expansions) introduce fresh mechanics like Tera Pokémon or reworked ex cards, keeping the game dynamic while staying true to its roots.
\item Casual Accessibility: Preconstructed decks (e.g., Battle Decks, League Battle Decks) provide an affordable, ready-to-play entry point, ideal for casual fun or learning the ropes.
\item Affordable Entry Point: Compared to MTG or Yu-Gi-Oh!, casual Pokémon play is budget-friendly with starter products, and Pokémon TCG Live offers a free digital option (though with in-game grind).
\item Cultural Impact: Tied to the Pokémon franchise’s massive popularity—spanning games, anime, and merchandise—the TCG taps into a nostalgic and vibrant universe, enhancing its appeal through familiar characters and lore.
\item Educational Benefits: The game teaches basic math (damage calculation), planning (energy management), and reading skills (card text), making it a subtle learning tool for younger players.
\item Replayability: Randomized draws, evolving board states, and matchup variety ensure games feel fresh, especially in casual settings with diverse Pokémon options.
\item Global Events: The Pokémon World Championships and regional tournaments offer a prestigious stage for competitive players, blending prizes with community celebration.
\item Art and Design: Pokémon cards are renowned for their stunning artwork—full-art cards, Secret Rares, and nostalgic reprints—making collecting as rewarding as playing.
\end{itemize}

Historically, the Pokémon TCG’s simplicity and Pokémon brand loyalty made it a gateway TCG, and today, it balances kid-friendly charm with enough depth to sustain a competitive scene.
