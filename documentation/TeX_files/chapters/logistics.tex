\chapter{Card Logistics}





\section{Card Types}

There are two primary categories of cards - unit cards (or units for short) and spell cards (or spells for short). Every card has a card `type' associated with it. Spell cards are differentiated in their function and purpose, but are simply a card with the type of `spell'. The unit cards encompass all of the remaining card types. Every unit card also has one or more sub-type(s) (also called secondary type(s)), which is a more focused type specific to the card itself (primarily the art/lore of the unit).

\subsubsection{Types}
The primary types, of which each card is assigned one, are listed below with a brief description.
\begin{description}
	\item[Earth:] Units aligned with nature, resilience, and the physical world. Earth units often have high defense and abilities that enhance survivability.
	\item[Water:] Units embodying fluidity, adaptability, and the power of the seas. Water units typically specialize in control effects and healing.
	\item[Fire:] Units representing destruction, passion, and chaotic power. Fire units are known for high attack values and aggressive effects.
	\item[Air:] Units characterized by speed, agility, and freedom. Air units often have evasion abilities and excel at quick, tactical strikes.
	\item[Electric:] Units aligned with energy, and the power of electricity. Electrical units often have abilities focused on stunning enemies, or interupting other cards.
	\item[Light:] Units symbolizing purity, order, and restoration. Light units frequently provide healing, support, and protective effects.
	\item[Nature:] Units embodying the wild, growth, and the harmony of living things. Nature units often excel in versatility, with abilities that promote regeneration, summon allies, or manipulate the battlefield with natural forces.
	\item[Dark:] Units associated with corruption, death, and forbidden power. Dark units excel at destruction, disruption, and sacrificial effects.
	\item[Spell:] Non-unit cards that represent magical powers, blessings, equipment, spell effects, artifacts, curses, or more. Spells provide a wide range of one-time or ongoing effects to influence the game.
\end{description}

\subsubsection{Sub-Types}
The secondary types, of which each unit card is assigned one or more, are listed below with a brief description.
\begin{description}
	\item[Avian:] Winged creatures of the sky, known for their agility, keen senses, and mastery of aerial maneuvers.
	\item[Dragon:] Powerful, mythical beings with immense strength and destructive capabilities.
	\item[Beast:] Natural creatures with raw physical power, often found in forests and plains.
	\item[Elemental:] Manifestations of elemental forces, bound to the magic of nature.
	\item[Aquatic:] Sea-dwelling creatures adept at controlling water and adapting to fluid environments.
	\item[Warrior:] Skilled fighters trained in physical combat, often acting as frontline attackers.
	\item[Spellcaster:] Mystics and mages who wield magical powers to cast spells and manipulate the battlefield.
	\item[Machine:] Artificial constructs built for labor, warfare, or magical purposes.
	\item[Ghost:] Ethereal spirits bound to the mortal realm, often with abilities to haunt or possess.
	\item[Insect:] Swarming, persistent creatures that thrive in large numbers and rapid reproduction.
	\item[Reptile:] Cold-blooded creatures known for their resilience and venomous strikes.
	\item[Fairy:] Small, magical beings capable of blessings, mischief, and subtle manipulations.
	\item[Undead:] Reanimated corpses and skeletal beings driven by dark magic or lingering souls.
	\item[Botanic:] Living plants and fungal creatures that can entangle foes, spread toxins, or regenerate.
\end{description}




\section{Cards}

\begin{figure}[h]
    \centering
    \includegraphics[width=\textwidth]{images/card_details.png} 
    \caption{This is a sample card with a brief explanation for what represents what on the card. This is a rank 1 water unit with a subtype of aquatic.}
    \label{fig:sample_card}
\end{figure}



\subsection{Unit Cards}

\textbf{Unit Cards:} Unit cards, or simply units, are one of the two primary card categories in the game, distinct from spell cards. Each unit card is assigned a single primary type - Earth, Water, Fire, etc - which defines its thematic alignment, such as high defense creatures for Earth or aggressive effects for Fire. Additionally, every unit possesses one or more subtypes (e.g., Dragon, Warrior, Beast), reflecting its specific identity, lore, and artwork, which often influence its effects. Units can be rank 1 to 5, and can be ranked up using any unit of the same type (exceptions apply when effects state otherwise) with a unit of one rank higher. Units feature attack and defense stats determined by their rank, starting at a base of 500 total points for Rank 1 and increasing by 500 points for each rank. They can be played to the upper row of the field (unit zone). Each unit card has two effects, activated by tapping the card, with tapping locking out the second effect until untapped by another card’s effect. Units can attack during the Battle Phase, regardless of their tapped state, targeting opposing units or the opponent directly if no units remain.



\subsection{Spell Cards}

\textbf{Spell Cards:} Spell cards, or simply spells, are one of the two primary card categories in the game, distinct from unit cards. Each spell card is assigned the primary type `Spell,' distinguishing it from unit types, and represents magical powers, blessings, equipment, artifacts, curses, or other effects. Unlike units, spells do not have subtypes, attack, or defense stats, focusing instead on their functional purpose. They can be played to the lower row of the field (spell zone). Spells range from rank 1 to rank 3 and can be ranked up just like any other card using a spell card of a higher rank. Each spell card has two effects, activated by tapping the card, with tapping locking out the second effect until untapped by another card’s effect. Spell effects can be activated immediately upon playing and at any point during either player’s turn, offering a wide range of one-time or ongoing influences, such as stat boosts, control, or disruption, often supporting units in the upper row. Rather than an attack and defense value, spells all include a stat bonus (e.g., +0, +5, -10) which is applied to the active unit card in the field zone above the spell. These stat bonuses vary heavily depending on the spell, but typically range from -10 to 10 for rank 1 spells, increasing or decreasing by 10 for each rank higher.


\subsection{Lore Cards}
// TODO - these are still in development







\section{Effect Variations}
Effects are the core mechanics that drive strategic interaction in the game, modifying the state of play through targeted actions and conditional triggers. Each effect is composed of three key elements: \textit{targets}, which define what the effect applies to; \textit{actions}, which specify what happens to those targets; and \textit{conditions}, which determine when or how the effect can be activated. Targets range from specific cards (e.g., "rank 2 units" or "card(s) under this card") to broad categories (e.g., "all cards" or "opponent’s cards"), allowing for precise or sweeping impacts. Actions encompass a variety of outcomes, such as destroying cards, drawing resources, or modifying stats, shaping the game’s flow and player decisions. Conditions impose requirements or timing, like discarding a card, losing points, or waiting for an attack, adding depth and tactical nuance. Together, these elements create a flexible system where effects can be combined to produce diverse and dynamic gameplay scenarios.


\subsection{Effect Types\label{CardEffectsSection}}

\begin{itemize}
    \item \textbf{Normal}: A Normal effect is one that can be activated during your turn by tapping the card.
    \item \textbf{Continuous}: This effect is activated like a normal effect, but stays active while the card is on the field.
    \begin{itemize}
		\item Continuous effects remain active after tapping; tapping only prevents the second effect’s activation.
		\item When a continuous effect is untapped, its effect is no longer active.
		\item Re-tapping can reactivate the continuous effect if chosen.
    \end{itemize}
	\item \textbf{Counter}: This effect can only be activated in response to a specific opponents action (determined by the card effect).
	\item \textbf{Equip}: This card is 'equip' to another card and remains active until the other card or this card is destroyed. 
	\begin{itemize}
		\item When a card is destroyed, all equip cards connected to that card are destroyed with it.
		\item If an equip card is ranked up, it is no longer considered active or equip to what it was before ranking up.
		\item You can equip a card to either players card, but the equip card always goes to the discard pile of the players who had the card in their deck at the start of the game.
	\end{itemize}
	\item Latent: This effect is triggered only when this card is currently untapped and meets some specific trigger condition. 
	\item \textbf{Dormant}: This is an effect that can only be activated when the card is tapped and can only be activated on a turn after the card was tapped.
	\item \textbf{Passive}: This effect is always active and is triggered whenever the effect conditions are met, whether tapped or untapped.
	\item \textbf{Overload:} This effect can be activated by tapping the card and discarding another card from your hand, offering a powerful one-time effect that exceeds typical rank limitations.
	\item \textbf{Echo:} This effect activates once when the card is played, then activates again the next time the card is untapped, tapping it only on the second activation.
	\item \textbf{Pulse}: This effect can be activated by tapping the card during your turn or your opponent’s turn whenever a game state changes or is about to change (e.g., a card is played, destroyed, tapped, untapped, a point is lost, etc), offering a quick, situational benefit (e.g., stat boost, draw a card, deal minor damage).
\end{itemize}

\subsection{Unit Effect Variations}
// TODO

\subsection{Spell Effect Variations}
// TODO


\subsection{Effect Conditions}
Many effects have a condition that must be met in order to use the effect or as part of the effect's actions. A list of these conditions follows with a brief description of what they mean.
\begin{description}
	\item["Discard one card to\dots":] The player must choose and discard one card from their hand to activate the effect of the card. Sometimes this condition is further specific, in which case the player must choose and discard one card matching the exact condition to activate the effect of the card. The  follow-up effect is considered activated after the action of discarding the card.
	\item["Discard this card to\dots":] The player must discard the specific card to trigger its effect, often a one-time cost for a powerful result. The  follow-up effect is considered activated after the action of discarding the card.
	\item["If you have\dots":] A condition based on whether the player has a specific card, type, or number of cards in play or in hand (e.g., "If you have a Fire unit on the field\dots").
	\item["Let your opponent\dots":] The player allows the opponent to perform a specific action or gain some benefit, often with a trade-off for the player. The follow-up effect is considered activated after the action done by the opponent and can only be activated if the opponent successfully takes that action.
	\item["Lose X point(s) to\dots":] The player must sacrifice a certain amount of points (e.g., life points, mana) to trigger the effect. The follow-up effect is considered activated after the player loses the point(s).
	\item["This turn\dots":] The effect lasts or can only be used within the current turn, typically providing a temporary advantage or ability.
	\item["While this card remains on the field\dots":] The effect remains active as long as the card stays in play, often providing continuous benefits or penalties.
	\item["When this card is send to the discard pile\dots":] This effect is triggered when the card is discarded or destroyed, usually offering a secondary benefit upon leaving the field. A card must be considered untapped to activate this effect. The follow-up effect is considered activated after the card is sent to the discard pile.
	\item["When another card is destroyed\dots":] This condition activates the effect when a different card is destroyed, potentially creating a chain reaction. The follow-up effect is considered activated after the other card is destroyed.
	\item["When an enemy card attacks\dots":] The effect activates when an enemy unit declares an attack, providing defensive actions, counterattacks, or interrupts. The follow-up effect is activated before the attack.
	\item["If you control\dots":] A condition based on controlling a specific unit or type of unit, or having a certain number of cards in play.
	\item["At the start of your turn\dots":] The effect triggers automatically at the beginning of the player's turn, typically with a passive or setup ability.
	\item["After this card attacks\dots":] The effect occurs after the card has attacked, often used for follow-up actions or penalties. The follow-up effect is activated after the attack is finished.
	\item["When this card attacks\dots":] The effect occurs after the card has attacked, often used for follow-up actions or penalties. The follow-up effect is activated before the attack.
	\item["During your opponent's phase\dots"]  The effect activates during the opponent's phase (the exact phase will be specified by the card), often to disrupt or delay their strategy.
	\item["When a card effect activates\dots"] The effect activates when another card effect activates. The follow-up effect occurs before the card effect that triggered this card. This is typically known as a counter effect.
	\item["Destroy one card to\dots"] The player must destroy a card they control to activate the effect. The follow-up effect activates after the destruction.
	\item["Destroy this card to\dots"] The player must destroy this card from the field to trigger its effect, often a one-time cost. The follow-up effect activates after destruction.
	\item["Skip your next turn to\dots"] The player must skip their next turn to activate the effect, a significant cost for a powerful outcome. The follow-up effect activates after the turn is skipped.
	\item["Tap\dots", "Untap\dots" ]
\end{description}

    

  
  


\subsection{Effect Targets}
Many effects have targets which the effect applies to. These targets can be broad or specific depending on the effect and all conditions of the target must be met in order for the effect to target that card. These are all pretty precise and straight forward. Regardless, a list of common target formats follows with a brief description of what they mean.
\begin{description}
	\item["one", "two", \dots] Often, an effect specifies a number of cards that its effect applies to. This is straight forward and the number must be adhered to unless another card specifies otherwise.
	\item["rank 1", "rank 2", \dots] Often, a specific rank of card is specified in the target. This must be a card matching that rank value.
	\item["rank x or lower"] This represents a card matching the rank or having a rank lower than the rank specified. 
	\item["rank x or higher"]  This represents a card matching the rank or having a rank higher than the rank specified. 
	\item["dark", "light", "spell", \dots] Often, a card will specify a type of card in the target. This type must be adhered to.
	\item["dragon", "warrior", \dots] Often, a card will specify a sub-type of card in the target. This type must be adhered to.
	\item["this card"] The effect targets the card it is currently associated with, often used for self-targeting abilities.
	\item["all cards", "each"] A broad target that includes all cards of a specified type or set. For example, "all Fire-type units" or "each unit on the field."
	\item["opponent's cards"] Specifies that the target is the opponent's cards, often used for disrupting or removing enemy units or spells.
	\item["your cards"] Specifies that the target is your own cards, often for protection or support abilities that affect your side of the field.
	\item["lowest attack", "highest defense", \dots] A conditional target based on the card’s stats. For example, "lowest attack" targets the card with the lowest attack value.
	\item["in your hand", "in your discard pile", "in your deck", \dots] Specifies cards in specific locations, such as those in the player's hand or discard pile.
	\item["equipped unit"] Targets a unit that is currently equipped with an item or equipment card.
	\item["another card on the field."] This would be any card on either player's side of the field. There must be another card on the field other than the card that says this to activate this effect.
	\item["another card on your side of the field."] This would be any card on your side of the field. There must be another card on the field other than the card that says this to activate this effect.
	\item["card(s) under this card"] This is a card that is physically underneath the card in question (not to be confused with \textit{below} the card on the field).
	\item["card(s) under another card"] Targets any card(s) physically underneath a different card on the field, distinct from the card with the effect.
	\item["card(s) below this card"] This is a card that is \textit{below} the card in questions position on the field - such as a spell card below a unit (not to be confused with \textit{under} a card).
\end{description}
The targets of a card can be any combination of the above descriptors or other possible card features.



\subsection{Effect Actions}
Effect actions define what an effect does to the targeted cards or entities. These actions are typically used in the effect’s resolution and determine how the game state is modified. Below is a list of common effect actions with a brief description of each:

\begin{description}
	\item[Destroy] The target card is removed from the field and sent to the discard pile. This action usually eliminates the target from play completely.  
	\item[Send] The target card is moved from one place to another, such as from the field to the discard pile, from hand to deck, or other designated zones depending on the effect.  
	\item[Mill] The target card or cards are placed from the top of a deck into the discard pile, often used to deplete the opponent's deck or trigger discard-related effects.  
	\item[Discard] The target card is taken from the player's hand and placed in the discard pile, typically reducing resources or disrupting the opponent’s strategy.  
	\item[Counter] The target card’s effect is negated or stopped entirely. A counter might prevent the resolution of a spell, effect, or action from the opponent, nullifying its effect.
	\item[Return] The target card is returned to its previous location or zone, such as returning a card from the field to the hand, deck, or another appropriate zone.  
	\item[Equip] A specific piece of equipment is attached to a target card, usually a unit, granting it additional abilities or buffs.  
	\item[Activate] The effect triggers or activates a specified ability or action on a card, often requiring specific conditions or costs to be met.  
	\item[Add] The effect adds a specified card or value to a target card, player, or zone. For example, adding cards to a player’s hand.  
	\item[Shuffle] The target cards are shuffled back into the deck, discard pile, or any other designated zone. This is often used to obscure the order of cards and introduce randomness.  
	\item[Reveal] The target card is revealed face-up to all players, usually as a means of providing information or triggering specific effects based on visibility.  
	\item[Search] The player may search through their deck, hand, for a specific card, typically allowing for strategic card retrieval.  
	\item[Gain] The target card or player gains a benefit, such as life points, attack points, defense points, or other advantages.  
	\item[Lose] The target card or player loses a specified amount of points, attack, defense, or other metric.  
	\item[Skip] The target player or card’s action is skipped, preventing them from performing a certain action, such as skipping a phase or a turn.
	\item[Draw] Moves cards from the top of the deck to the hand, increasing available resources.
	\item[Play] Places a card into active use on the field from a zone (e.g., hand, discard pile).
	\item[Place] Positions a card in a specific location (e.g., under another card).
	\item[Swap] Exchanges the position of two cards or swaps their stats (e.g., Attack and Defense).
	\item[Prevent] Stops a specified action from activating or affecting a target (e.g., preventing destruction).
	\item[Modify] Alters a card’s stats, rank, or type (e.g., increasing rank or doubling Attack).
	\item[Rank Up] Stacking a card of one higher rank on top of a card.
\end{description}















\section{Serial Number Generation for Trading Cards}
\label{sec:serial_number_generation}

The serial number (SN) for each trading card is a concise, unique identifier generated from the card's attributes, designed to distinguish potentially millions of unique cards while maintaining a compact format of characters. This section details the algorithm implemented in the Python function \texttt{generate\_serial\_number}, which leverages card attributes such as name, type, subtypes, level, attack, defense, effects, styles, and rarity. This method is found in the \texttt{card\_maker\_ui.py} script (see \ref{sec:card_maker_ui}).

The base $N$ is a set of characters which are chosen for their unique appearance with the selected font. Some characters clash with others and it is hard to tell which ones are which. Some characters also cause the positioning of the serial number to chance (based on how the font display is programmed). For The characters chosen in $N$, both of these were considered and undesired characters were removed. The important part is that a unique serial number can be generated with a sufficiently large $N$. At the time of writing this, the selected characters are (subject to change) as follows:

\begin{quotation}
	0123456789AaBbCcDdEeFfGHhiKkLMmNnOoPRrSsTtUuVvWwXxYZz
\end{quotation}

\subsection{Overview of the Serial Number Format}
The serial number is constructed as a string by concatenating encoded representations of the card's attributes in a fixed order. Each component is derived as follows:
\begin{itemize}
	\item \textbf{Initial Character}: The first letter of the card's name, capitalized (e.g., ``D'' for ``Dragon'').
	\item \textbf{Level}: The card's level as a single digit (e.g., ``5'' for level 5).
	\item \textbf{Type and Subtypes ID}: A base-$N$ encoded index representing the combination of the card's type and subtypes, derived from a precomputed list of valid combinations. A maximum total of 6 total types can be input.
	\item \textbf{Attack and Defense IDs}: Single characters in base-$N$ representing the attack and defense values, scaled relative to the card's level.
	\item \textbf{Effect IDs}: The first character of each effect text (or ``0'' if no effect).
	\item \textbf{Effect Style IDs}: Base-$N$ encoded indices of the effect styles from a predefined list.
	\item \textbf{Rarity}: A single integer representing the card's rarity (e.g., ``0'' for base rarity).
	\item \textbf{Padding}: A final ``0'' character reserved for future expansion.
\end{itemize}
The resulting SN is a string such as ``U40XY87S060900'', where each segment encodes specific card data.

\subsection{Attribute Encoding Details}
\subsubsection{Type and Subtypes Combination ID}
The type and subtypes are combined into a comma-separated string (e.g., ``Fire, Dragon, Warrior''), which is then matched against a precomputed list of all valid combinations generated by \texttt{get\_sequence\_combinations}. This function:
\begin{itemize}
	\item Takes a list of possible types (e.g., \texttt{ALL\_TYPES\_LIST\_LOWER}) and generates all unique combinations, ensuring at most one primary type (e.g., ``Fire'') per combination.
	\item Caches results in \texttt{ALL\_SEQUENCE\_BUFFER} to avoid recomputation.
	\item Returns a sorted list of combinations, such as [``'', ``Dragon'', ``Fire'', ``Fire, Dragon'', \ldots].
\end{itemize}
The \texttt{get\_combination\_id} function finds the index of the input combination in this list and converts it to a fixed-length base-$N$ string using the character set $C = \{0, 1, \ldots, 9, A, \ldots, Z\}$, where $|C| = N$. For an index $i$, the base-$N$ representation is computed as:
\begin{align}
	\text{result} = \bigoplus_{k} C[i \mod N], \quad i = \lfloor i / N \rfloor,
\end{align}
where $\bigoplus$ denotes string concatenation from right to left, and the result is padded with leading zeros to 4 digits (e.g., index 10 becomes ``000A'').

\subsubsection{Attack and Defense IDs}
The attack and defense values (integers from 0 to $500 \times \text{level}$) are mapped to one of $N$ segments using \texttt{get\_number\_id}. The maximum value is $M = 500 \times \text{level}$, and the segment size is:
\begin{align}
S = \frac{M}{N}.
\end{align}
The segment index is calculated as:
\begin{align}
	\text{index} = \frac{N}{M}\max\left(0, \left\lfloor \min(\text{value}, M) \right\rfloor\right),
\end{align}
and mapped to a character in $C$ (e.g., 1500 at level 5 might map to ``K'').

\subsubsection{Effect and Style IDs}
For effects, the first character is used (e.g., ``D'' for ``Draw 1 card''), or ``0'' if the effect is empty. Effect styles are encoded via \texttt{get\_index\_in\_baseN}, which finds the index of the style in \texttt{VALID\_OVERLAY\_STYLES} and converts it to base-$N$ (e.g., index 1 becomes ``1'', $N$ becomes ``10'').

\subsection{Uniqueness and Conciseness}
Uniqueness is achieved by:
\begin{itemize}
	\item Using a deterministic combination of all card attributes.
	\item Leveraging the large combinatorial space of base-$N$ encoding (e.g., if $N=36$, then $36^4 = 1,679,616$ possible type/subtype IDs).
	\item Including level-specific scaling for attack and defense, reducing overlap across levels.
\end{itemize}
The SN length is kept concise (currently set to 15 characters) by using single characters where possible and fixed-length encoding for variable components.

\subsection{Preprocessing and Performance}
The \texttt{preprocess\_all\_types\_list} function precomputes type combinations in a separate thread, storing them in \texttt{ALL\_SEQUENCE\_BUFFER}. The global flag \texttt{PREPROCESSING\_FINISHED} ensures the SN generation waits for this step, returning ``Loading..'' if incomplete. This system balances uniqueness, readability, and efficiency for large-scale card generation.





