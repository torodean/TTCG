\newcommand{\andspace}[1]{\hspace{#1}\textrm{and}\hspace{#1}}

\numberwithin{equation}{section}
\setlength{\columnsep}{.5cm}
\setlength{\columnseprule}{1pt}
\def\columnseprulecolor{\color{black}}

\newcommand{\abs}[1]{\left| #1 \right|}
\newcommand{\inner}[1]{\langle #1 \rangle}
\newcommand{\norm}[1]{\left\lVert#1\right\rVert}
\newcommand{\spanvect}{\textnormal{span}}
\newcommand{\union}{\cup}
\newcommand{\Union}{\bigcup}

%create a section without making the section title.
\newcommand\invisiblesection[1]{%
	\refstepcounter{section}%
	\addcontentsline{toc}{section}{\protect\numberline{\thesection}#1}%
	\sectionmark{#1}}

%Makes a chapter with no title
\makeatletter
\newcommand{\unchapter}[1]{%
	\begingroup
	\let\@makechapterhead\@gobble % make \@makechapterhead do nothing
	\chapter{#1}
	\endgroup
}
\makeatother

% Define the questions environment
\newenvironment{questions}{%
	\begin{tcolorbox}[
		colback=orange!10,
		colframe=orange!50,
		arc=0pt,
		boxrule=1pt,
		title={\faQuestion\ Questions},
		fonttitle=\bfseries,
		coltitle=blue!50!black,
		colbacktitle=yellow!20,
		attach title to upper={\par},
		boxsep=0pt,
		]
		\begin{itemize}  % Start the itemize environment
			\BODY  % This will contain the list of questions
		\end{itemize}  % End the itemize environment
	\end{tcolorbox}
}

% define interesting note
\newenvironment{interestnote}{
	\begin{tcolorbox}[
		colback=green!10,  % Very light green background color of the box
		colframe=green!50,  % Border color of the box
		arc=0pt,  % Adjust the corner radius of the box
		boxrule=1pt,  % Border thickness
		title={\faLightbulb\ Interesting Note},  % The lightbulb icon and title
		fonttitle=\bfseries,  % Font style for the title
		coltitle=blue!50!black,  % Color for the title text
		colbacktitle=yellow!20,  % Background color for the title
		attach title to upper={\par},
		boxsep=0pt,  % Adjust the space between the content and the box
		]
	}{
	\end{tcolorbox}
}

\definecolor{lightpurple}{RGB}{220,180,240}

\newenvironment{quotationbox}{
	\begin{tcolorbox}[
		colback=lightpurple!10,  % Very light purple background color of the box
		colframe=lightpurple!50,  % Border color of the box
		arc=0pt,  % Adjust the corner radius of the box
		boxrule=1pt,  % Border thickness
		title={\faQuoteLeft\ Quotation},  % The quotation icon and title
		fonttitle=\bfseries,  % Font style for the title
		coltitle=blue!50!black,  % Color for the title text
		colbacktitle=yellow!20,  % Background color for the title
		attach title to upper={\par},
		boxsep=0pt,  % Adjust the space between the content and the box
		]
	}{
	\end{tcolorbox}
}

% define a URL
\newenvironment{urlbox}{
	\begin{tcolorbox}[
		colback=blue!10,  % Very light green background color of the box
		colframe=blue!50,  % Border color of the box
		arc=0pt,  % Adjust the corner radius of the box
		boxrule=1pt,  % Border thickness
		title={URL},  % The lightbulb icon and title
		fonttitle=\bfseries,  % Font style for the title
		coltitle=blue!50!black,  % Color for the title text
		colbacktitle=yellow!20,  % Background color for the title
		attach title to upper={\par},
		boxsep=0pt,  % Adjust the space between the content and the box
		]
	}{
	\end{tcolorbox}
}

\newcommand{\unfinished}{%
	\par\noindent%
	\setlength{\fboxsep}{10pt} % Adjust the padding
	\fcolorbox{red}{red!20}{%
		\begin{minipage}{\dimexpr\linewidth-2\fboxsep}%
			\vspace{5pt} % Adjust the vertical spacing
			\centering
			\textcolor{red}{\textbf{\faExclamation\ Section in Progress\ \faExclamation}}
			\vspace{5pt} % Adjust the vertical spacing
		\end{minipage}%
	}%
	\par
}

% Define the custom terminal style
\lstdefinestyle{terminalstyle}{
    language=bash,                   % Set the language to Bash
    basicstyle=\ttfamily\color{white}, % Use a monospaced font and white text
    backgroundcolor=\color{black},   % Background color
    frame=tb,                        % Top and bottom frame lines
    framerule=0.5pt,                 % Frame rule width
    xleftmargin=5pt,                 % Left margin
    xrightmargin=0pt,                % Right margin
    rulecolor=\color{black!20},      % Frame color
    showstringspaces=false,          % Don't show spaces in strings
    upquote=true,                    % Use straight quotes
    commentstyle=\color{green},      % Comments are green
    keywordstyle=\color{orange!60},  % Keywords
    numbers=left,                    % Line numbers on the left
    numberstyle=\tiny\color{black!90},   % Line number style
    captionpos=b,                    % Caption position (bottom)
    breaklines=true,                 % Automatically break long lines
    postbreak=\mbox{\textcolor{green}{$\hookrightarrow$}\space}, % Line continuation symbol
}

% Define a bash/shell style.
\lstdefinestyle{shellstyle}{
	language=sh,                    % Set the language to Shell
	basicstyle=\ttfamily,           % Use a monospaced font
	backgroundcolor=\color{gray!10},% Background color
	keywordstyle=\color{blue},      % Keywords are blue
	commentstyle=\color{green!40!black}, % Comments are green
	stringstyle=\color{red},        % Strings are red
        rulecolor=\color{black},        % Frame color
	breakatwhitespace=false,        % Don't break lines at whitespace
	breaklines=true,                % Automatically break long lines
	postbreak=\mbox{\textcolor{red}{$\hookrightarrow$}\space}, % Line continuation symbol
	showstringspaces=false,         % Don't show spaces in strings
	captionpos=b,                   % Caption position (bottom)
	frame=single,                   % Add a frame around listings
	numbers=left,                   % Line numbers on the left
	numberstyle=\tiny\color{gray},  % Line number style
	stepnumber=1,                   % Line number increments
	tabsize=4,                      % Tab size
	xleftmargin=5pt,                % Left margin
	xrightmargin=0pt                % Right margin
}

% Define Python style
\lstdefinestyle{pythonstyle}{
	language=Python,
	basicstyle=\ttfamily\small,
	numbers=left,
	numberstyle=\tiny\color{gray},
	frame=single,
	rulecolor=\color{orange},
        backgroundcolor=\color{orange!05},
	keywordstyle=\color{blue},
	commentstyle=\color{green!40!black},
	stringstyle=\color{red},
	breaklines=true,
	tabsize=4,
	captionpos=b,
	extendedchars=true,
	inputencoding=utf8,
	showspaces=false,
	showstringspaces=false,
	xleftmargin=5pt,                % Left margin
	xrightmargin=0pt                % Right margin
}


% Define C++ style
\lstdefinestyle{cppstyle}{
	language=C++,
	basicstyle=\ttfamily\small,
	numbers=left,
	numberstyle=\tiny\color{gray},
	frame=single,
	rulecolor=\color{black!30},
	backgroundcolor=\color{gray!05},% Background color
	keywordstyle=\color{blue},
	commentstyle=\color{green!40!black},
	stringstyle=\color{red},
	breaklines=true,
	tabsize=4,
	captionpos=b,
	extendedchars=true,
	inputencoding=utf8,
	showspaces=false,
	showstringspaces=false,
	xleftmargin=5pt,                % Left margin
	xrightmargin=0pt                % Right margin
}

% Define dark green color
\definecolor{darkgreen}{RGB}{0,100,0}

% Define Makefile style
\lstdefinestyle{makestyle}{
	language=make,
	basicstyle=\ttfamily\small,
	numbers=left,
	numberstyle=\tiny\color{gray},
	frame=single,
	backgroundcolor=\color{darkgreen!05},
	rulecolor=\color{darkgreen},
	keywordstyle=\color{blue},
	commentstyle=\color{green!40!black},
	stringstyle=\color{red},
	breaklines=true,
	tabsize=4,
	captionpos=b,
	extendedchars=true,
	inputencoding=utf8,
	showspaces=false,
	showstringspaces=false,
	xleftmargin=5pt,                % Left margin
	xrightmargin=0pt                % Right margin
}

% Define Git style
\lstdefinestyle{gitstyle}{
	language=bash, % Assuming Git commands are written in a bash-like language
	basicstyle=\ttfamily\small,
	numbers=left,
	numberstyle=\tiny\color{gray},
	frame=single,
	backgroundcolor=\color{blue!05},
	rulecolor=\color{blue},
	keywordstyle=\color{blue},
	commentstyle=\color{green!40!black},
	stringstyle=\color{red},
	breaklines=true,
	tabsize=4,
	captionpos=b,
	extendedchars=true,
	inputencoding=utf8,
	showspaces=false,
	showstringspaces=false,
	xleftmargin=5pt,                % Left margin
	xrightmargin=0pt                % Right margin
}

% Define C# style
\lstdefinestyle{csharpstyle}{
	language=[Sharp]C,
	basicstyle=\ttfamily\small,
	numbers=left,
	numberstyle=\tiny\color{gray},
	frame=single,
	rulecolor=\color{purple},
	backgroundcolor=\color{purple!05},
	keywordstyle=\color{blue},
	commentstyle=\color{green!40!black},
	stringstyle=\color{red},
	breaklines=true,
	tabsize=4,
	captionpos=b,
	extendedchars=true,
	inputencoding=utf8,
	showspaces=false,
	showstringspaces=false,
	xleftmargin=5pt,                % Left margin
	xrightmargin=0pt                % Right margin
}

% Define Java style
\lstdefinestyle{javastyle}{
	language=Java,
	basicstyle=\ttfamily,
	keywordstyle=\color{blue},
	rulecolor=\color{red!80},
	backgroundcolor=\color{red!05},
	commentstyle=\color{green!70!black},
	stringstyle=\color{red},
	numbers=left,
	numberstyle=\tiny\color{gray},
	stepnumber=1,
	breaklines=true,
	breakatwhitespace=false,
	tabsize=4,
	frame=single,
	showspaces=false,
	showstringspaces=false,
	xleftmargin=5pt,                % Left margin
	xrightmargin=0pt                % Right margin
}

% Define JavaScript language settings
\lstdefinelanguage{JavaScript}{
	keywords={typeof, new, true, false, catch, function, return, null, catch, switch, var, if, in, while, do, else, case, break, default},
	keywordstyle=\color{blue}\bfseries,
	ndkeywords={class, export, boolean, throw, implements, import, this},
	ndkeywordstyle=\color{darkgray}\bfseries,
	identifierstyle=\color{black},
	sensitive=false,
	comment=[l]{//},
	morecomment=[s]{/*}{*/},
	commentstyle=\color{purple}\ttfamily,
	stringstyle=\color{red}\ttfamily,
	morestring=[b]',
	morestring=[b]",
}

% Define JavaScript style
\lstdefinestyle{jsstyle}{
	language=JavaScript,
	basicstyle=\ttfamily,
	rulecolor=\color{red!80},
	backgroundcolor=\color{red!05},
	keywordstyle=\color{blue},
	commentstyle=\color{green!70!black},
	stringstyle=\color{red},
	numbers=left,
	numberstyle=\tiny\color{gray},
	stepnumber=1,
	breaklines=true,
	breakatwhitespace=false,
	tabsize=4,
	frame=single,
	showspaces=false,
	showstringspaces=false,
	xleftmargin=5pt,                % Left margin
	xrightmargin=0pt                % Right margin
}

% Define HTML style
\lstdefinestyle{htmlstyle}{
	language=HTML,
	basicstyle=\ttfamily\small,
	numbers=left,
	numberstyle=\tiny\color{gray},
	frame=single,
	rulecolor=\color{black},
	backgroundcolor=\color{yellow!05},
	keywordstyle=\color{blue},
	commentstyle=\color{green!40!black},
	stringstyle=\color{red},
	breaklines=true,
	tabsize=4,
	captionpos=b,
	extendedchars=true,
	inputencoding=utf8,
	showspaces=false,
	showstringspaces=false,
	xleftmargin=5pt,                % Left margin
	xrightmargin=0pt                % Right margin
}

% Define CSS style
\lstdefinestyle{cssstyle}{
	language=HTML,
	basicstyle=\ttfamily\small,
	numbers=left,
	numberstyle=\tiny\color{gray},
	frame=single,
	rulecolor=\color{black},
	keywordstyle=\color{blue},
	commentstyle=\color{green!40!black},
	stringstyle=\color{red},
	breaklines=true,
	tabsize=4,
	captionpos=b,
	extendedchars=true,
	inputencoding=utf8,
	showspaces=false,
	showstringspaces=false,
	xleftmargin=5pt,                % Left margin
	xrightmargin=0pt                % Right margin
}

% Define LaTeX style
\lstdefinestyle{latexstyle}{
	language=TeX,
	basicstyle=\ttfamily\small,
	numbers=left,
	numberstyle=\tiny\color{gray},
	frame=single,
	rulecolor=\color{brown},
	backgroundcolor=\color{brown!05},
	keywordstyle=\color{blue},
	commentstyle=\color{green!40!black},
	stringstyle=\color{red},
	breaklines=true,
	tabsize=4,
	captionpos=b,
	extendedchars=true,
	inputencoding=utf8,
	showspaces=false,
	showstringspaces=false,
	xleftmargin=5pt,                % Left margin
	xrightmargin=0pt                % Right margin
}

% Define PHP style
\lstdefinestyle{phpstyle}{
	language=PHP,
	basicstyle=\ttfamily\small,
	numbers=left,
	numberstyle=\tiny\color{gray},
	frame=single,
	rulecolor=\color{black},
	keywordstyle=\color{blue},
	commentstyle=\color{green!40!black},
	stringstyle=\color{red},
	breaklines=true,
	tabsize=4,
	captionpos=b,
	extendedchars=true,
	inputencoding=utf8,
	showspaces=false,
	showstringspaces=false,
	xleftmargin=5pt,                % Left margin
	xrightmargin=0pt                % Right margin
}
